\documentclass[11pt, oneside]{article}   	% use "amsart" instead of "article" for AMSLaTeX format
\usepackage{geometry}                		% See geometry.pdf to learn the layout options. There are lots
\geometry{letterpaper}                   		% ... or a4paper or a5paper or ... 
%\geometry{landscape}                		% Activate for rotated page geometry
\usepackage[parfill]{parskip}    		% Activate to begin paragraphs with an empty line rather than an indent
\usepackage{graphicx}				% Use pdf, png, jpg, or eps§ with pdflatex; use eps in DVI mode
								% TeX will automatically convert eps --> pdf in pdflatex		
\usepackage{amssymb}
\linespread{1}

\title{Physics Test 2 Notes}
\author{Cole Cuthbert}
\date{11/3/15}	
\begin{document}
\maketitle

Momentum: \hspace{.5cm} ${\vec{p}=m\vec{v}}$ \hspace{.5cm} ${<F_{Net}>}={\Delta\vec{p}\over\Delta\vec{t}}$\

\hspace{.5cm}\emph{Impulse:} \hspace{.5cm} $\Delta\vec{p}$\\

Kinetic Energy: \hspace{.5cm} $KE={pv\over2}$ \hspace{.5cm} $KE={mv^2\over2}$\hspace{.5cm} $\Delta KE=KE_2-KE_1$\

Work: \hspace{.5cm} ${W}={F\Delta x}$\

Power: \hspace{.5cm} $P={W\over\Delta t}$ \hspace{.5cm} $P=\vec{F}*\vec{v}$\

Universal Gravitational Work and Potential Energy: \hspace{.5cm} $PE=-{{Gm_1m_2}\over{r^2}}$\

Gravitational Work and Potential Energy: \hspace{.5cm} $PE=mgy$\

Spring Force: \hspace{.5cm} ${PE=}{{kx^2} \over 2}$ \

Friction: \hspace{.5cm} $F_{Friction}=-\mu_k$\\

Circular Motion:\

\hspace{.25cm}\emph{Radial:} \hspace{.5cm} $F_{net}=ma_{centripetal}$ \hspace{1cm} $mg\cos\theta-F_n=ma_{centripetal}$ \hspace{1cm} $a_{centripetal}={v^2\over r}$\\

Collision: \hspace{.5cm} $P_f=P_i$\

\hspace{.25cm}\emph{Inelastic} \hspace{.5cm} $V_1f=V_2f=V_f$ \hspace{1cm} $m_{1i}v_{1i}+m_{2i}v_{2i}=m_{1f}v_{1f}+m_{2f}v_{2f}$\\
Final velocity of inelastic collision can be found through ${v_f={{m_1v_1+m_2v_2}\over{m_1+m_2}}}$\

\hspace{.25cm}\emph{Elastic} \hspace{.5cm} $KE_f=KE_i$\\
Center of mass can be found by using the same equation as final velocity for inelastic collisions ${v_f={{m_1v_1+m_2v_2}\over{m_1+m_2}}=v_{center of mass}}$
\end{document}  